\section{Regresionsanalyse?}\label{sec:regresionsanalyse}
Regressionsanalyse er en statistisk metode, der anvendes til at undersøge og modellere sammenhæng mllem en afhængig variabel og en uafhængig variabel. Denne metode er udbredt inden for områder som økonomi, sunhed- og samfunds forskning samt naturvidenskaben. Historisk set blev beregningerne udført manuelt, hvilket var både tidskrævende og tilbøjeligt til at give fejl. Den frøste version af regresionsanalyse blev for første gang nævnt i en artikel af Francis Galton i år 1886. Francis Galton var en engelsk videnskabsmand. Francis Galton var blandt andet kendt for at være opfinderen af fingeraftryks identificering. Den metode som Francis Galton beskrev, er i dag kendt som Regression mod gennemsnittet. Begræbet beskriver at hvis en variabel er langt fra gennemsnittet, så vil den næste værdi af samme variabel være tættere på gennemsnittet. Galton målte højden på 250 forældre og deres 930 voksene børn. Han justerede møderenes højde ved at gange med 1,08 Forældernes højde var morens og farens højde delt med to. Hr. Galton sikere sig at der ikke vare en systemmatisk tendenst til at høje mænd giftede sig med høje kvinder og at lave mænd gitede sig med lave kvinder. Plotte han sine obeservationer ind i et kordinatsystem hvor han hen af x-aksen havde Forældernes højde og ad y-aksen havde han børnenes højde. Galton troede, han havde gjort sig en stor opdagelse. Han fandt ud af at sønner af meget høje fædre ofte var højere end gemmensnittet, men dog stadig lavere end deres fædre. Det så ud som om at der var en ukendt faktor, der var skyld i at menneskets højde bevlgede sig væk fra det ekstremt høje ned til det mere normale gennemsnit. Det fænomen kaldte han ”regression towards mediocrity” (”tilbagevenden til middel-mådelighed”) % KILDE: https://lex.dk/regression_mod_gennemsnittet og https://lru.praxis.dk/Lru/microsites/hvadermatematik/hem3download/kap8_QR5b_Galton_Regression_towards_the_mean.pdf
Selvom at Galtons opdagelse var banebryende og introducerede begræbet regression, anvendes denne metode ikke længere. Hans tilgang var basseret på specefikke observationer, så som netop højde, dette betød at man manglede den matematiske tilgang. Galtons bidrag dannede dog fundament for videreudviklingen af regressionsmetoder. I dag bygger regressionsanalyse på et stærkt matematisk og statistisk fundament. Dette fundament blev blandt andet bygget af Carl Fredrich Gauss og Adrien.Marie Legendre, der introducerede mindste kvardraters metode. Denne metode, spiller stadig en central rolle i regressionsanalyser. Der er mange fordele ved at anvende regresionsanalyse, men der er dog også en del ulemper ved det. Metoden forudsætter at dine data kan beskrives som en funktion, at dine data reelt 
% Hvad er Mindste Kvadraters Metode? Hvad kan den anvendes til? (Også praktisk!) Hvad er dens begrænsninger? (Historisk, hvad med gamle dage)
\subsection{Matematikken bag Mindste Kvadraters Metode}
% Redegør for matematikken bag Mindste Kvadraters Metode

\subsubsection{Funktioner med to variabler}
Du har måske stiftet bekendskab med funktioner af en variabel. Disse funktioner er ofte skrevet som \begin{math}f(x) = x^2\end{math}.. Her er ønsker man at finde værdien $f(x)$ eller $g(x)$ for et givet $x$ værdi. Her giver man altså en værdi og for en værdi tilbage. Men hvad nu. hvis du skal beskrive en sammmenhæng, der er afhængig af to variabler? Her kommer funktioner med to variabler ind i billedet. Når der arbejdes med funktioner af en variabel forgår dette i et koordinatsystem med en x-akse og en y-akse. Altså et to dimentionelt koordinatsystem. For at kunne afbillede funktioner af to variable, arbejdes der ikke længere i et koordinatsystem med to dimentioner, men i et tre dimentionelt koordinatsystem. Her er der en x-akse, en y-akse og en z-akse. Den nye akse (z-aksen) står vinkelret på de to andre akser.\\ For at bedre forstå hvordan det 3 dimentionelle koorinatsystem hænger sammen kan man tænke på det som en kasse. Hvor x-aksen er længden, y-aksen er bredden og z-aksen er højden. (\cite[246-248]{funktionrAfToVariable}) Når der arbejdes med funktioner af to variable vil den typiske notation se således ud: $f(x,y)$ her er både $x$ og $y$ variabler. Disse to variable kan bruges til at beskrive et punkt i det 3 dimentionelle koordinatsystem. Dette skyldes at $z = f(x,y)$ Et konkret eksempel kunne være et vejrkort. Her er temparaturen afhængig af to variable. Her angiver $x$ og $y$ en lokation og $z$ angiver temparaturen. På den måde afgænger temparaturen af to variable, det geografiske punkt som beskrives $(x,y)$.\\


% Redegør for funktioner med to variabler (Forklar forskellen mellem en- og to-dimensionelle funktioner) Introduktion til funktioner af to variable.

\subsubsection{Partielle afledede}
% Redegør for de partielle afledede

\subsection{Anvendelse af Mindste Kvadraters Metode}
% Vis hvordan Mindste Kvadraters Metode kan anvendes på et selvvalgt datasæt (Husk en fortolkning af resultaterne)


\section{Implementering af Mindste Kvadraters Metode}
% Anvend pseudokode til at vise hvordan Mindste Kvadraters Metode kan implementeres


\subsection{Program Design}
% Forklar hvilke overvejelser der er gjort i forhold til program design, sprogvalg, biblioteker, etc. Teorisk baggrund!!!

\subsection{Program redegørsel}
% Redegør for programmet og dets funktioner (Husk at inkludere kodeeksempler)

\section{Fordele og Begrænsninger}
%Her analyseres metoden på et bredere niveau ved at vurdere dens styrker og svagheder i en programmeringssammenhæng.

\section{Diskussion af Kodningsmetoder}
% While loops, for loops, rekursive funktioner, ect.
