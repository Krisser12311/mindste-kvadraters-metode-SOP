\section*{Resume}
Denne opgave undersøger anvendelsen af Mindste Kvadraters Metode til at modellere data og ved hjælp af det, finde den bedst passende funktion. Metoden er baseret på matematikken bag funktioner af to variable og partiel differentiation. Partiel differentiation bruges her til at finde et minimumspunkt hvilket gør det muligt at finde det sted hvor summen af kvadraternes areal mellem datapunkter og en lineær linje er mindst. Den teoretiske forståelse blev omsat til praksis ved anvendelse på et datasæt og implementering i Python. I programmeringskonteksten blev metoden implementeret med fokus på modularitet og brugervenlighed, en pseudokode blev lavet for at strukturere processen. Fordelene ved programmering inkluderer hurtigere beregninger, automatisering og muligheden for at håndtere store datasæt, men udfordringer som fejl i input og beregnings hastighed ved store datasæt bliver også belyst. En diskussionen om kodningsmetoder fremhævede forskellene mellem procedural, objektorienterede og funktionel programmerings tilgange, hvor valget af metode afhænger af projektets kompleksitet og skalerbarhed. Der blev desuden også diskuteret hvilke kontrolstrukture der er bedst i hvilke senarier. Samlet set demonstrerer opgaven, hvordan Mindste Kvadraters Metode i kombination med programmering kan omsætte komplekse matematiske beregninger til effektive og anvendelige løsninger i dataanalyse. \newpage